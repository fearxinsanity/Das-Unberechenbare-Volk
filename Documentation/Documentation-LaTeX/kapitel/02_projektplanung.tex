% ============================================================
% KAPITEL 2: PROJEKTPLANUNG
% ============================================================

\section{Projektplanung}

\subsection{Vorgehensmodell}

Für die Durchführung dieses Projekts wurde das Wasserfallmodell als lineares, sequenzielles Vorgehensmodell gewählt. Diese Entscheidung basiert auf dem festen Abgabetermin am 06.02.2026 und dem klar definierten, unveränderlichen Anforderungskatalog gemäß Projektauftrag.

Das Wasserfallmodell sieht vor, dass die Entwicklung in aufeinanderfolgenden, klar abgegrenzten Phasen erfolgt: Anforderungsanalyse und Planung, Entwurf und Design, Implementierung, Testen und Qualitätssicherung sowie Dokumentation und Abnahme. Jede Phase wird erst nach vollständigem Abschluss der vorhergehenden Phase begonnen. Eine detaillierte Zeitplanung mit Meilensteinen ist im Anhang A2 (Gantt-Diagramm) dargestellt.

Dieser strukturierte Ansatz ermöglicht eine präzise Planung und Kontrolle des Projektfortschritts sowie eine umfassende Dokumentation jeder Entwicklungsphase entsprechend der \ac{IHK}-Anforderungen. Zudem minimiert das Modell das Risiko einer schleichenden Umfangserweiterung (Scope Creep), da alle Anforderungen zu Projektbeginn festgelegt werden.

Im Vergleich zu agilen Vorgehensmodellen wie Scrum, die auf iterative Entwicklung und flexible Anforderungen ausgelegt sind, bietet das Wasserfallmodell für dieses Projekt mit stabilen Vorgaben die bessere Eignung. Die Risiken des Wasserfallmodells, insbesondere die späte Fehlererkennung, werden durch eine umfassende Planungsphase und systematische Qualitätssicherung minimiert.


\subsection{Ressourcen- und Ablaufplanung}

\subsubsection{Zeitplanung}

Die Projektdurchführung erstreckt sich über einen Zeitraum von 23 Wochen (06.09.2025 bis 06.02.2026) und ist in fünf Hauptphasen unterteilt, die sequenziell nach dem Wasserfallmodell ablaufen. Die folgende Tabelle gibt einen Überblick über die zeitliche Verteilung der Projektphasen:

\begin{table}[H]
    \centering
    \caption{Zeitliche Verteilung der Projektphasen}
    \label{tab:zeitplanung}
    \begin{tabularx}{\textwidth}{|l|X|l|}
        \hline
        \textbf{Epic} & \textbf{Beschreibung} & \textbf{Zeitraum} \\
        \hline
        Epic 1 & Projektplanung (Mockups, UML, Spezifikation, Risikoanalyse) & Sep. 2025 \\
        \hline
        Epic 2 & Technische Umsetzung (Implementierung: Zufallselemente, Kernkomponenten, Backend, UI) & Okt. -- Nov. 2025 \\
        \hline
        Epic 3 & Qualitätssicherung (Testkonzept, Validierung, Testdurchführung, Fehlerbehebung) & Dez. 2025 -- Jan. 2026 \\
        \hline
        Epic 4 & Projektabschluss (Dokumentation, Ausführbare Datei, Präsentation) & Jan. -- Feb. 2026 \\
        \hline
    \end{tabularx}
\end{table}

Eine detaillierte Visualisierung der Zeitplanung mit allen Arbeitspaketen, Abhängigkeiten und Meilensteinen ist im Anhang A2, Abbildung~\ref{fig:gantt} als Gantt-Diagramm dargestellt. Die Epics sind in einzelne User Stories unterteilt, die nacheinander bzw. teilweise parallel abgearbeitet werden.


\subsubsection{Kostenplanung}

Die Kostenplanung basiert auf einer Kalkulation der Personalkosten sowie der benötigten Sachmittel. Für die Projektdurchführung wird von einem Gesamtaufwand von 120 Arbeitsstunden ausgegangen, verteilt auf die vier Projekt-Epics. Bei einem Stundensatz von 6,50€ ergibt sich folgende Kostenübersicht:

\begin{table}[H]
    \centering
    \caption{Kostenplanung}
    \label{tab:kostenplanung}
    \begin{tabularx}{\textwidth}{|l|X|r|r|r|}
        \hline
        \textbf{Kostenart} & \textbf{Beschreibung} & \textbf{Menge/h} & \textbf{Einzelpreis} & \textbf{Gesamt} \\
        \hline
        Personalkosten & Entwicklung \& Dokumentation & 120h & 6,50€ & 780,00€ \\
        \hline
        \multicolumn{5}{|l|}{\textbf{Sachmittelkosten}} \\
        \hline
        -- Software & JDK 25, JavaFX, Maven, IntelliJ IDEA UE & -- & 0,00€ & 0,00€ \\
        \hline
        -- Hardware & Entwicklungsrechner (vorhanden) & -- & 0,00€ & 0,00€ \\
        \hline
        -- Lizenzen & Microsoft Office 365 (vorhanden) & -- & 0,00€ & 0,00€ \\
        \hline
        \multicolumn{4}{|r|}{\textbf{Gesamtkosten}} & \textbf{780,00€} \\
        \hline
    \end{tabularx}
\end{table}


\subsection{Risikoanalyse}

Die Risikoanalyse identifiziert potenzielle Gefährdungen für den Projekterfolg und definiert präventive sowie reaktive Maßnahmen zur Risikominimierung. Die Risiken werden nach ihrer Eintrittswahrscheinlichkeit (gering, mittel, hoch) und ihren Auswirkungen auf Projekttermin, -kosten oder -qualität bewertet. Die folgende Übersicht zeigt die identifizierten Risiken mit entsprechenden Gegenmaßnahmen:

\begin{table}[H]
    \centering
    \caption{Risikoauflistung}
    \label{tab:risiken}
    \small
    \begin{tabularx}{\textwidth}{|X|c|c|X|}
        \hline
        \textbf{Risiko} & \textbf{Wahrsch.} & \textbf{Auswirk.} & \textbf{Prävention/Reaktive Maßnahmen} \\
        \hline
        Performance-Probleme bei hoher Wähleranzahl & Mittel & Hoch & Frühzeitige Belastungstests, Optimierung der Datenstrukturen, Profiling \\
        \hline
        Kompatibilitätsprobleme auf Schulrechnern & Gering & Hoch & Klärung IT-Infrastruktur, Test auf Schulrechner, jpackage mit JRE \\
        \hline
        Fehlerhafte Zufallsverteilungen & Mittel & Mittel & Statistische Validierung, Unit-Tests, Vergleich mit Referenzimplementierungen \\
        \hline
        Zeitverzug durch unterschätzte Komplexität & Mittel & Hoch & Pufferzeiten eingeplant, wöchentliche Fortschrittskontrolle, Priorisierung \\
        \hline
        Krankheitsbedingte Ausfälle & Gering & Mittel & Zeitpuffer, frühzeitige Information, Fokus auf Kernfunktionen \\
        \hline
        Unzureichende Dokumentation & Gering & Mittel & Kontinuierliche Dokumentation, frühzeitiges Review durch Betreuer \\
        \hline
        Nichterfüllung der IHK-Anforderungen & Gering & Hoch & Checkliste erstellen, regelmäßiger Abgleich mit Projektauftrag \\
        \hline
    \end{tabularx}
\end{table}
