% ============================================================
% ANHANG
% ============================================================

\subsection*{A1 UML-Diagramme}

% --- USE-CASE-DIAGRAMM ---
\begin{figure}[H]
    \centering
    % Ersetze "usecase.png" mit deinem Dateinamen
    % \includegraphics[width=0.8\textwidth]{bilder/usecase.png}
    
    [Hier Bild einfügen: usecase.png]
    
    \caption{Anwendungsfalldiagramm des Wahlsystems}
    \label{fig:usecase}
\end{figure}

% --- KLASSENDIAGRAMM ---
% \begin{figure}[H]
%     \centering
%     \includegraphics[width=\textwidth]{bilder/klassendiagramm.png}
%     \caption{Klassendiagramm}
%     \label{fig:klassendiagramm}
% \end{figure}

% --- SEQUENZDIAGRAMM ---
% \begin{figure}[H]
%     \centering
%     \includegraphics[width=\textwidth]{bilder/sequenzdiagramm.png}
%     \caption{Sequenzdiagramm: Datenfluss}
%     \label{fig:sequenz}
% \end{figure}


\newpage
\subsection*{A2 Zeitplanung}

\begin{figure}[H]
    \centering
    % Ersetze "gantt.png" mit deinem Dateinamen
    % \includegraphics[width=\textwidth]{bilder/gantt.png}
    
    [Hier Bild einfügen: gantt.png]
    
    \caption{Gantt-Diagramm}
    \label{fig:gantt}
\end{figure}


\newpage
\subsection*{A3 Screenshots der Anwendung}

% --- SCREENSHOTS ---
% \begin{figure}[H]
%     \centering
%     \includegraphics[width=0.9\textwidth]{bilder/screenshot_main.png}
%     \caption{Hauptansicht der Anwendung}
%     \label{fig:screenshot_main}
% \end{figure}

[Hier Screenshots der fertigen Anwendung einfügen]


\newpage
\subsection*{A4 Quellcode-Auszüge}

% Falls du wichtige Code-Ausschnitte zeigen willst:

% \begin{lstlisting}[language=Java, caption={Kernmethode der Simulation}]
% public void runSimulation() {
%     while (simulationRunning) {
%         calculateVoterMovement();
%         applyRandomEvents();
%         updateStatistics();
%         notifyObservers();
%     }
% }
% \end{lstlisting}

[Hier wichtige Code-Auszüge einfügen]
