% ============================================================
% KAPITEL 1: EINLEITUNG
% ============================================================

\section{Einleitung}

Im Rahmen der Ausbildung zum Fachinformatiker für Anwendungsentwicklung erfolgt ein auf IHK-Abschlussprojektniveau angestrebtes Simulationsprojekt. Es dient zur Überprüfung des Wissen- und Ausbildungsstands sowie zur Vorbereitung auf die Abschlussprüfung.

\subsection{Projektbeschreibung}

Die Unterrichtsstoffvermittlung komplexer politischer Wahlsysteme in Schulen erfolgt aktuell überwiegend durch statische Lernmittel und lineare Medien, was zu einem reduzierten Verständnis der dynamischen Zusammenhänge einer Wahl und ein verringertes Lerninteresse für die Lernenden führt. Diese Lernmethoden sollen durch eine interaktive Desktop-Anwendung ergänzt werden, um den Ansatz \glqq Learning by Doing\grqq{} zu implementieren.

\subsection{Projektumfeld}

Die Projektdurchführung erfolgt als schulische Projektarbeit direkt beim Auftraggeber, der \glqq Beruflichen Schule der Hanse- und Universitätsstadt Rostock -Technik-\grqq{}. Die Zielplattform der Anwendung bilden die schuleigenen Windows Client.

\subsection{Projektziele}

Das Ziel des Projektes ist es, die bereits vorhandene Unterrichtsstoffvermittlung mit einer Desktopanwendung zum Analysieren von dynamischen Wählerwanderungen und externen Einflüssen zu erweitern.

Die Benutzeroberfläche wird mittels JavaFX, einem MVC-basierten Java Framework, realisiert und mit FXML, einer XML-basierten Sprache definiert. Die Gestaltung erfolgt über CSS.

Die Entwicklung des Backends erfolgt mit Java als Programmiersprache. Als Schnittstelle zwischen Backend und Frontend wird eine Controller-Klasse implementiert, welche mit der GUI kommuniziert.

\subsection{Projektbegründung}

Durch das Projekt sollen nicht sichtbare Prozesse, wie die Wählerwanderung oder externe Einflüsse visualisiert werden. Zudem soll damit die didaktische Qualität gesteigert werden, indem die Lernenden beispielsweise bei einer Parameteränderung direktes Feedback erhalten.

\subsection{Projektschnittstellen}

\subsubsection{Technische Schnittstellen}

Da die Software als Standalone-Anwendung konzipiert ist, existieren keine Schnittstellen zu vorhandenen IT-Infrastrukturen oder Datenbanken. 

\subsubsection{Organisatorische Schnittstellen}

Der fachliche Ansprechpartner des Kunden ist Herr Patett als Hauptverantwortlicher.

\subsection{Projektabgrenzung}

Durch den begrenzten Zeitumfang des Projektes wurden bestimmte Eingrenzungen getroffen:

\begin{itemize}
    \item Keine Bereitstellung für weitere Betriebssysteme. Der Fokus liegt auf Windows. Weitere Systeme werden erst nach Finalisierung des Projekts realisiert.
    \item Es werden keine Simulationsdaten in einer Datenbank gespeichert. Eine weitere Verwendung dieser historischen Daten ist nicht vorgesehen.
    \item Eine Implementierung einer \glqq künstlichen Intelligenz \grqq{} erfolgt nicht. Eine Implementierung würde den Projektumfang übersteigen.
    \item Es werden keine realen historischen Wahldaten verwendet. Die Simulation würde damit auf historische Szenarien festgelegt werden.
\end{itemize}

