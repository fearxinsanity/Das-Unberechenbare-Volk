% ============================================================
% KAPITEL 4: QUALITÄTSSICHERUNG
% ============================================================

\section{Qualitätssicherung}

\subsection{Testkonzept}

% TODO: Hier das Testkonzept beschreiben
[Hier beschreiben: Welche Testarten werden eingesetzt? (Unit-Tests, Integrationstests, Systemtests, Usability-Tests)]

% Beispiel für eine Testübersicht:
% \begin{table}[H]
%     \centering
%     \caption{Übersicht der Testarten}
%     \begin{tabularx}{\textwidth}{|l|X|l|}
%         \hline
%         \textbf{Testart} & \textbf{Beschreibung} & \textbf{Werkzeug} \\
%         \hline
%         Unit-Tests & Test einzelner Methoden und Klassen & JUnit 5 \\
%         \hline
%         Integrationstests & Test der Komponenteninteraktion & JUnit 5 \\
%         \hline
%         Systemtests & Test des Gesamtsystems & Manuell \\
%         \hline
%     \end{tabularx}
% \end{table}


\subsection{Testdurchführung}

% TODO: Hier die Testdurchführung beschreiben
[Hier beschreiben: Welche Tests wurden durchgeführt? Was waren die Ergebnisse?]


\subsection{Validierung der Zufallsmodelle}

% TODO: Hier die Validierung beschreiben
[Hier beschreiben: Wie wurde sichergestellt, dass die Zufallsverteilungen korrekt implementiert sind?]


\subsection{Soll-Ist-Vergleich}

% TODO: Hier den Soll-Ist-Vergleich einfügen
[Hier beschreiben: Vergleich der geplanten mit den tatsächlichen Zeiten]

% Beispiel für Soll-Ist-Vergleich:
% \begin{table}[H]
%     \centering
%     \caption{Soll-Ist-Vergleich der Projektphasen}
%     \begin{tabularx}{\textwidth}{|X|r|r|r|}
%         \hline
%         \textbf{Phase} & \textbf{Soll (h)} & \textbf{Ist (h)} & \textbf{Differenz} \\
%         \hline
%         Projektplanung & 20 & 22 & +2 \\
%         \hline
%         Implementierung & 60 & 58 & -2 \\
%         \hline
%         Test & 25 & 25 & 0 \\
%         \hline
%         Dokumentation & 15 & 15 & 0 \\
%         \hline
%         \textbf{Gesamt} & \textbf{120} & \textbf{120} & \textbf{0} \\
%         \hline
%     \end{tabularx}
% \end{table}
