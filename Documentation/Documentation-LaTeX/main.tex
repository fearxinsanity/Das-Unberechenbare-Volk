% ============================================================
% HAUPTDATEI - IHK Projektdokumentation
% "Das Unberechenbare Volk" - Wahlsimulation
% ============================================================

\documentclass[12pt, a4paper]{article}

% === PAKETE ===
\usepackage[utf8]{inputenc}
\usepackage[T1]{fontenc}
\usepackage[ngerman]{babel}
\usepackage{geometry}
\usepackage{setspace}
\usepackage{graphicx}
\usepackage{hyperref}
\usepackage{listings}
\usepackage{xcolor}
\usepackage{booktabs}
\usepackage{tabularx}
\usepackage{parskip}
\usepackage{fancyhdr}
\usepackage{acronym}
\usepackage{caption}
\usepackage{float}
\usepackage{pdfpages}
\usepackage[backend=biber, style=authoryear]{biblatex}
\usepackage{titlesec}
\usepackage{tocloft}

% === SEITENRÄNDER ===
\geometry{
    left=3.5cm,
    right=3cm,
    top=3.5cm,
    bottom=3cm
}

% === ZEILENABSTAND ===
\onehalfspacing

% ============================================================
% GLIEDERUNG NACH IHK-RICHTLINIEN
% - Dezimalsystem mit arabischen Ziffern
% - Kein Punkt nach der letzten Ziffer
% - Maximal 3 Ebenen (z.B. 1.1.1)
% - Arabische Seitenzahlen ab Abkürzungsverzeichnis (Seite 1)
% - Römische Seitenzahlen für Anhang (beginnend bei I)
% ============================================================

% Nummerierung ohne Punkt am Ende: "1" statt "1."
\renewcommand{\thesection}{\arabic{section}}
\renewcommand{\thesubsection}{\thesection.\arabic{subsection}}
\renewcommand{\thesubsubsection}{\thesubsection.\arabic{subsubsection}}

% Section-Format: "1 Einleitung"
\titleformat{\section}
  {\normalfont\Large\bfseries}{\thesection}{1em}{}

% Subsection-Format: "1.1 Unterkapitel"
\titleformat{\subsection}
  {\normalfont\large\bfseries}{\thesubsection}{1em}{}

% Subsubsection-Format: "1.1.1 Unter-Unterkapitel"
\titleformat{\subsubsection}
  {\normalfont\normalsize\bfseries}{\thesubsubsection}{1em}{}

% Inhaltsverzeichnis: Ohne Punkt nach Nummern
\renewcommand{\cftsecaftersnum}{}
\renewcommand{\cftsubsecaftersnum}{}
\renewcommand{\cftsubsubsecaftersnum}{}

% ============================================================

% === KOPF- UND FUSSZEILE ===
\pagestyle{fancy}
\fancyhf{}
\fancyhead[R]{\leftmark}
\fancyfoot[C]{\thepage}
\renewcommand{\headrulewidth}{0.4pt}

% === HYPERLINKS ===
\hypersetup{
    colorlinks=true,
    linkcolor=black,
    urlcolor=blue,
    citecolor=black
}

% === CODE-DARSTELLUNG ===
\definecolor{codegreen}{rgb}{0,0.6,0}
\definecolor{codegray}{rgb}{0.5,0.5,0.5}
\definecolor{codepurple}{rgb}{0.58,0,0.82}
\definecolor{backcolour}{rgb}{0.95,0.95,0.92}

\lstdefinestyle{mystyle}{
    backgroundcolor=\color{backcolour},
    commentstyle=\color{codegreen},
    keywordstyle=\color{magenta},
    numberstyle=\tiny\color{codegray},
    stringstyle=\color{codepurple},
    basicstyle=\ttfamily\footnotesize,
    breakatwhitespace=false,
    breaklines=true,
    captionpos=b,
    keepspaces=true,
    numbers=left,
    numbersep=5pt,
    showspaces=false,
    showstringspaces=false,
    showtabs=false,
    tabsize=2,
    frame=single
}
\lstset{style=mystyle}

% === LITERATURVERZEICHNIS ===
% Erstelle eine Datei "literatur.bib" für deine Quellen
% \addbibresource{literatur.bib}

% ============================================================
% MEINE DATEN
% ============================================================
\newcommand{\Name}{Nico Hoffmann}
\newcommand{\ProjektTitel}{Das Unberechenbare Volk -- Entwicklung einer JavaFX-Anwendung zur Simulation von Wählerverhalten}
\newcommand{\Betrieb}{SIV.AG}
\newcommand{\Ausbildungsberuf}{Fachinformatiker für Anwendungsentwicklung}
\newcommand{\IHK}{IHK zu Rostock}
\newcommand{\Abgabedatum}{06.02.2026}

% ============================================================
% DOKUMENT BEGINNT
% ============================================================
\begin{document}

% ############################################################
% IHK-REIHENFOLGE:
% 1. Deckblatt
% 2. Bestätigter Projektantrag
% 3. Persönliche Erklärung
% 4. Inhaltsverzeichnis
% 5. Textteil (inkl. Tabellen, Bilder, technische Darstellungen)
% 6. Literaturverzeichnis/Quellenangaben
% 7. Anlagen (römische Nummerierung ab I)
% ############################################################

% ============================================================
% KEINE SEITENZAHLEN bis zum Abkürzungsverzeichnis
% ============================================================
\pagenumbering{gobble}

% ============================================================
% 1. DECKBLATT
% ============================================================
\begin{titlepage}
    \centering
    \vspace*{1cm}
    
    {\Large \textbf{Projektdokumentation}}\\[0.5cm]
    {\large zur Prüfungsvorbereitung}\\[0.3cm]
    {\large \Ausbildungsberuf}\\[2cm]
    
    {\LARGE \textbf{\ProjektTitel}}\\[0.5cm]
    {\large Wie beeinflussen externe Faktoren das Endergebnis einer demokratischen Wahl?}\\[2cm]
    
    \begin{tabular}{ll}
        \textbf{Auszubildender:} & \Name \\[0.3cm]
        \textbf{Ausbildungsbetrieb:} & \Betrieb \\[0.3cm]
        \textbf{Prüfungsausschuss:} & \IHK \\[0.3cm]
        \textbf{Abgabedatum:} & \Abgabedatum \\
    \end{tabular}
    
    \vfill
\end{titlepage}

% ============================================================
% 2. BESTÄTIGTER PROJEKTANTRAG
% ============================================================
% Füge hier deinen genehmigten Projektantrag als PDF ein:
% \includepdf[pages=-]{projektantrag.pdf}

% Falls du keinen PDF hast, nutze diesen Platzhalter:
\newpage
\begin{center}
    \vspace*{5cm}
    {\Large \textbf{Bestätigter Projektantrag}}\\[2cm]
    {\large [Hier den genehmigten Projektantrag einfügen]}\\[1cm]
    {\small Zum Einfügen als PDF: Entkommentiere die Zeile}\\
    {\small \texttt{\textbackslash includepdf[pages=-]\{projektantrag.pdf\}}}
\end{center}

% ============================================================
% 3. PERSÖNLICHE ERKLÄRUNG (Erklärung des Prüfungsteilnehmers)
% ============================================================
\newpage

\begin{center}
    {\Large \textbf{Erklärung des Prüfungsteilnehmers}}
\end{center}

\vspace{1cm}

Ich versichere hiermit, dass ich die vorliegende Projektarbeit selbstständig verfasst und keine anderen als die angegebenen Quellen und Hilfsmittel benutzt habe. Die Stellen der Arbeit, die anderen Werken dem Wortlaut oder dem Sinn nach entnommen wurden, sind in jedem Fall unter Angabe der Quelle als Entlehnung kenntlich gemacht.

\vspace{1cm}

Diese Arbeit hat in gleicher oder ähnlicher Form noch keiner Prüfungsbehörde vorgelegen.

\vspace{1cm}

Ich bin damit einverstanden, dass diese Projektarbeit zu Prüfungszwecken eingesehen und verwendet werden darf.

\vspace{3cm}

\noindent
\begin{tabular}{p{6cm}p{1cm}p{6cm}}
\hrulefill & & \hrulefill \\
Ort, Datum & & Unterschrift Prüfungsteilnehmer
\end{tabular}

\vspace{2cm}

\begin{center}
    {\large \textbf{Erklärung des Ausbildungsbetriebes}}
\end{center}

\vspace{0.5cm}

Die oben beschriebene Projektarbeit wurde im Ausbildungsbetrieb durchgeführt.

\vspace{2cm}

\noindent
\begin{tabular}{p{6cm}p{1cm}p{6cm}}
\hrulefill & & \hrulefill \\
Ort, Datum & & Unterschrift Ausbildungsbetrieb
\end{tabular}

% ============================================================
% 4. INHALTSVERZEICHNIS (ohne Seitenzahl)
% ============================================================
\newpage
\tableofcontents
\newpage

% ============================================================
% 5. TEXTTEIL
% ============================================================

% ============================================================
% SEITENNUMMERIERUNG: ARABISCH ab hier, beginnend bei 1
% ============================================================
\pagenumbering{arabic}

% --- Abkürzungsverzeichnis (Seite 1) ---
\section*{Abkürzungsverzeichnis}
\addcontentsline{toc}{section}{Abkürzungsverzeichnis}

\begin{acronym}[XXXXX]
    \acro{API}{Application Programming Interface}
    \acro{GUI}{Graphical User Interface}
    \acro{IHK}{Industrie- und Handelskammer}
    \acro{ISO}{International Organization for Standardization}
    \acro{JDK}{Java Development Kit}
    \acro{LTS}{Long Term Support}
    \acro{MVC}{Model-View-Controller}
    \acro{SE}{Standard Edition}
    \acro{UI}{User Interface}
    \acro{UML}{Unified Modeling Language}
    \acro{UX}{User Experience}
\end{acronym}
\newpage

% --- Abbildungsverzeichnis ---
\listoffigures
\addcontentsline{toc}{section}{Abbildungsverzeichnis}
\newpage

% --- Tabellenverzeichnis ---
\listoftables
\addcontentsline{toc}{section}{Tabellenverzeichnis}
\newpage

% --- Kapitel ---
% ============================================================
% KAPITEL 1: EINLEITUNG
% ============================================================

\section{Einleitung}

Im Rahmen der Ausbildung zum Fachinformatiker für Anwendungsentwicklung erfolgt ein auf IHK-Abschlussprojektniveau angestrebtes Simulationsprojekt. Es dient zur Überprüfung des Wissen- und Ausbildungsstands sowie zur Vorbereitung auf die Abschlussprüfung.

\subsection{Projektbeschreibung}

Die Unterrichtsstoffvermittlung komplexer politischer Wahlsysteme in Schulen erfolgt aktuell überwiegend durch statische Lernmittel und lineare Medien, was zu einem reduzierten Verständnis der dynamischen Zusammenhänge einer Wahl und ein verringertes Lerninteresse für die Lernenden führt. Diese Lernmethoden sollen durch eine interaktive Desktop-Anwendung ergänzt werden, um den Ansatz \glqq Learning by Doing\grqq{} zu implementieren.

\subsection{Projektumfeld}

Die Projektdurchführung erfolgt als schulische Projektarbeit direkt beim Auftraggeber, der \glqq Beruflichen Schule der Hanse- und Universitätsstadt Rostock -Technik-\grqq{}. Die Zielplattform der Anwendung bilden die schuleigenen Windows Client.

\subsection{Projektziele}

Das Ziel des Projektes ist es, die bereits vorhandene Unterrichtsstoffvermittlung mit einer Desktopanwendung zum Analysieren von dynamischen Wählerwanderungen und externen Einflüssen zu erweitern.

Die Benutzeroberfläche wird mittels JavaFX, einem MVC-basierten Java Framework, realisiert und mit FXML, einer XML-basierten Sprache definiert. Die Gestaltung erfolgt über CSS.

Die Entwicklung des Backends erfolgt mit Java als Programmiersprache. Als Schnittstelle zwischen Backend und Frontend wird eine Controller-Klasse implementiert, welche mit der GUI kommuniziert.

\subsection{Projektbegründung}

Durch das Projekt sollen nicht sichtbare Prozesse, wie die Wählerwanderung oder externe Einflüsse visualisiert werden. Zudem soll damit die didaktische Qualität gesteigert werden, indem die Lernenden beispielsweise bei einer Parameteränderung direktes Feedback erhalten.

\subsection{Projektschnittstellen}

\subsubsection{Technische Schnittstellen}

Da die Software als Standalone-Anwendung konzipiert ist, existieren keine Schnittstellen zu vorhandenen IT-Infrastrukturen oder Datenbanken. 

\subsubsection{Organisatorische Schnittstellen}

Der fachliche Ansprechpartner des Kunden ist Herr Patett als Hauptverantwortlicher.

\subsection{Projektabgrenzung}

Durch den begrenzten Zeitumfang des Projektes wurden bestimmte Eingrenzungen getroffen:

\begin{itemize}
    \item Keine Bereitstellung für weitere Betriebssysteme. Der Fokus liegt auf Windows. Weitere Systeme werden erst nach Finalisierung des Projekts realisiert.
    \item Es werden keine Simulationsdaten in einer Datenbank gespeichert. Eine weitere Verwendung dieser historischen Daten ist nicht vorgesehen.
    \item Eine Implementierung einer \glqq künstlichen Intelligenz \grqq{} erfolgt nicht. Eine Implementierung würde den Projektumfang übersteigen.
    \item Es werden keine realen historischen Wahldaten verwendet. Die Simulation würde damit auf historische Szenarien festgelegt werden.
\end{itemize}


\newpage

% ============================================================
% KAPITEL 2: PROJEKTPLANUNG
% ============================================================

\section{Projektplanung}

\subsection{Vorgehensmodell}

Für die Durchführung dieses Projekts wurde das Wasserfallmodell als lineares, sequenzielles Vorgehensmodell gewählt. Diese Entscheidung basiert auf dem festen Abgabetermin am 06.02.2026 und dem klar definierten, unveränderlichen Anforderungskatalog gemäß Projektauftrag.

Das Wasserfallmodell sieht vor, dass die Entwicklung in aufeinanderfolgenden, klar abgegrenzten Phasen erfolgt: Anforderungsanalyse und Planung, Entwurf und Design, Implementierung, Testen und Qualitätssicherung sowie Dokumentation und Abnahme. Jede Phase wird erst nach vollständigem Abschluss der vorhergehenden Phase begonnen. Eine detaillierte Zeitplanung mit Meilensteinen ist im Anhang A2 (Gantt-Diagramm) dargestellt.

Dieser strukturierte Ansatz ermöglicht eine präzise Planung und Kontrolle des Projektfortschritts sowie eine umfassende Dokumentation jeder Entwicklungsphase entsprechend der \ac{IHK}-Anforderungen. Zudem minimiert das Modell das Risiko einer schleichenden Umfangserweiterung (Scope Creep), da alle Anforderungen zu Projektbeginn festgelegt werden.

Im Vergleich zu agilen Vorgehensmodellen wie Scrum, die auf iterative Entwicklung und flexible Anforderungen ausgelegt sind, bietet das Wasserfallmodell für dieses Projekt mit stabilen Vorgaben die bessere Eignung. Die Risiken des Wasserfallmodells, insbesondere die späte Fehlererkennung, werden durch eine umfassende Planungsphase und systematische Qualitätssicherung minimiert.


\subsection{Ressourcen- und Ablaufplanung}

\subsubsection{Zeitplanung}

Die Projektdurchführung erstreckt sich über einen Zeitraum von 23 Wochen (06.09.2025 bis 06.02.2026) und ist in fünf Hauptphasen unterteilt, die sequenziell nach dem Wasserfallmodell ablaufen. Die folgende Tabelle gibt einen Überblick über die zeitliche Verteilung der Projektphasen:

\begin{table}[H]
    \centering
    \caption{Zeitliche Verteilung der Projektphasen}
    \label{tab:zeitplanung}
    \begin{tabularx}{\textwidth}{|l|X|l|}
        \hline
        \textbf{Epic} & \textbf{Beschreibung} & \textbf{Zeitraum} \\
        \hline
        Epic 1 & Projektplanung (Mockups, UML, Spezifikation, Risikoanalyse) & Sep. 2025 \\
        \hline
        Epic 2 & Technische Umsetzung (Implementierung: Zufallselemente, Kernkomponenten, Backend, UI) & Okt. -- Nov. 2025 \\
        \hline
        Epic 3 & Qualitätssicherung (Testkonzept, Validierung, Testdurchführung, Fehlerbehebung) & Dez. 2025 -- Jan. 2026 \\
        \hline
        Epic 4 & Projektabschluss (Dokumentation, Ausführbare Datei, Präsentation) & Jan. -- Feb. 2026 \\
        \hline
    \end{tabularx}
\end{table}

Eine detaillierte Visualisierung der Zeitplanung mit allen Arbeitspaketen, Abhängigkeiten und Meilensteinen ist im Anhang A2, Abbildung~\ref{fig:gantt} als Gantt-Diagramm dargestellt. Die Epics sind in einzelne User Stories unterteilt, die nacheinander bzw. teilweise parallel abgearbeitet werden.


\subsubsection{Kostenplanung}

Die Kostenplanung basiert auf einer Kalkulation der Personalkosten sowie der benötigten Sachmittel. Für die Projektdurchführung wird von einem Gesamtaufwand von 120 Arbeitsstunden ausgegangen, verteilt auf die vier Projekt-Epics. Bei einem Stundensatz von 6,50€ ergibt sich folgende Kostenübersicht:

\begin{table}[H]
    \centering
    \caption{Kostenplanung}
    \label{tab:kostenplanung}
    \begin{tabularx}{\textwidth}{|l|X|r|r|r|}
        \hline
        \textbf{Kostenart} & \textbf{Beschreibung} & \textbf{Menge/h} & \textbf{Einzelpreis} & \textbf{Gesamt} \\
        \hline
        Personalkosten & Entwicklung \& Dokumentation & 120h & 6,50€ & 780,00€ \\
        \hline
        \multicolumn{5}{|l|}{\textbf{Sachmittelkosten}} \\
        \hline
        -- Software & JDK 25, JavaFX, Maven, IntelliJ IDEA UE & -- & 0,00€ & 0,00€ \\
        \hline
        -- Hardware & Entwicklungsrechner (vorhanden) & -- & 0,00€ & 0,00€ \\
        \hline
        -- Lizenzen & Microsoft Office 365 (vorhanden) & -- & 0,00€ & 0,00€ \\
        \hline
        \multicolumn{4}{|r|}{\textbf{Gesamtkosten}} & \textbf{780,00€} \\
        \hline
    \end{tabularx}
\end{table}


\subsection{Risikoanalyse}

Die Risikoanalyse identifiziert potenzielle Gefährdungen für den Projekterfolg und definiert präventive sowie reaktive Maßnahmen zur Risikominimierung. Die Risiken werden nach ihrer Eintrittswahrscheinlichkeit (gering, mittel, hoch) und ihren Auswirkungen auf Projekttermin, -kosten oder -qualität bewertet. Die folgende Übersicht zeigt die identifizierten Risiken mit entsprechenden Gegenmaßnahmen:

\begin{table}[H]
    \centering
    \caption{Risikoauflistung}
    \label{tab:risiken}
    \small
    \begin{tabularx}{\textwidth}{|X|c|c|X|}
        \hline
        \textbf{Risiko} & \textbf{Wahrsch.} & \textbf{Auswirk.} & \textbf{Prävention/Reaktive Maßnahmen} \\
        \hline
        Performance-Probleme bei hoher Wähleranzahl & Mittel & Hoch & Frühzeitige Belastungstests, Optimierung der Datenstrukturen, Profiling \\
        \hline
        Kompatibilitätsprobleme auf Schulrechnern & Gering & Hoch & Klärung IT-Infrastruktur, Test auf Schulrechner, jpackage mit JRE \\
        \hline
        Fehlerhafte Zufallsverteilungen & Mittel & Mittel & Statistische Validierung, Unit-Tests, Vergleich mit Referenzimplementierungen \\
        \hline
        Zeitverzug durch unterschätzte Komplexität & Mittel & Hoch & Pufferzeiten eingeplant, wöchentliche Fortschrittskontrolle, Priorisierung \\
        \hline
        Krankheitsbedingte Ausfälle & Gering & Mittel & Zeitpuffer, frühzeitige Information, Fokus auf Kernfunktionen \\
        \hline
        Unzureichende Dokumentation & Gering & Mittel & Kontinuierliche Dokumentation, frühzeitiges Review durch Betreuer \\
        \hline
        Nichterfüllung der IHK-Anforderungen & Gering & Hoch & Checkliste erstellen, regelmäßiger Abgleich mit Projektauftrag \\
        \hline
    \end{tabularx}
\end{table}

\newpage

% ============================================================
% KAPITEL 3: TECHNISCHE UMSETZUNG
% ============================================================

\section{Technische Umsetzung}

\subsection{Architektur und Design}

\subsubsection{Architektur-Konzept (MVC)}

Das Projekt folgt einer Model-View-Controller (\ac{MVC})-Architektur. Dieses Entwurfsmuster wurde gewählt, um eine klare Trennung der Verantwortlichkeiten zu gewährleisten und die Wartbarkeit sowie Erweiterbarkeit des Systems zu erhöhen.

Das \textbf{Model} enthält die gesamte Simulationslogik und ist vollständig vom User Interface entkoppelt. Die \textbf{View} ist ausschließlich für die Darstellung der Daten und die Erfassung von Benutzereingaben verantwortlich und wird mit JavaFX umgesetzt. Der \textbf{Controller} dient als Vermittler zwischen Model und View, verarbeitet Benutzereingaben und aktualisiert beide Komponenten entsprechend.

Als Alternative wurde das Model-View-View-Model (MVVM)-Pattern in Betracht gezogen, das durch Data Binding eine noch stärkere Entkopplung zwischen View und Logik ermöglicht. MVVM wird häufig in modernen \ac{UI}-Frameworks wie JavaFX eingesetzt und würde die automatische Synchronisation von Daten und \ac{UI}-Elementen vereinfachen. Allerdings wurde \ac{MVC} bevorzugt, da es für die Projektkomplexität ausreichend ist, eine geringere Einarbeitungszeit erfordert und eine klarere, explizite Kontrolle über den Datenfluss bietet. Der zusätzliche Overhead durch Data Binding würde bei diesem Projektumfang keinen signifikanten Mehrwert bieten.

Diese Architektur bietet den Vorteil, dass Änderungen an der Benutzeroberfläche die Simulationslogik nicht beeinträchtigen und umgekehrt. Zudem ermöglicht die klare Struktur eine parallele Entwicklung der Komponenten und erleichtert das Testen einzelner Module.


\subsubsection{Datenfluss}

Der Datenfluss im System folgt einem unidirektionalen Prinzip und ist im Sequenzdiagramm (siehe Anhang A1, Abbildung~\ref{fig:sequenz}) visualisiert. Die Daten durchlaufen drei zentrale Phasen: Eingabe, Verarbeitung und Ausgabe.

In der \textbf{Eingabephase} erfasst die View Konfigurationsparameter vom Benutzer, wie die Anzahl der Wähler und Parteien, Anfangspräferenzen, Medieneinfluss und Kampagnenbudgets. Diese Daten werden vom Controller validiert und als strukturierte Parameter an das Model übergeben.

Das Model führt in der \textbf{Verarbeitungsphase} die Simulation in diskreten Zeitschritten durch. Dabei werden die Eingabeparameter mit den Zufallsverteilungen (Normal-, Gleich-, Exponentialverteilung) kombiniert, um Meinungsänderungen der Wähler, Kampagneneffekte und zufällige Ereignisse zu berechnen. Das Ergebnis sind aktualisierte Zustandsdaten, die die Verteilung der Wählerpräferenzen, Unterstützerzahlen der Parteien und aufgetretene Ereignisse beschreiben.

In der \textbf{Ausgabephase} werden die berechneten Daten vom Controller abgerufen und an die View weitergegeben. Diese transformiert die numerischen Daten in visuelle Darstellungen (dynamische Diagramme, Animation) und textuelles Feedback (Ereignis-Feed). Die Aktualisierung erfolgt in Echtzeit, sodass Änderungen unmittelbar für den Benutzer sichtbar werden.

Dieser unidirektionale Datenfluss stellt sicher, dass die Komponenten lose gekoppelt bleiben und Daten niemals direkt von der View zum Model oder umgekehrt fließen, sondern immer über den Controller vermittelt werden.


\subsection{Auswahl der Technologien}

\subsubsection{Software- und Hardware}

Für die Entwicklung des Projekts wird eine Reihe von Software-Tools eingesetzt, die den Vorgaben der Kostenplanung entsprechen und größtenteils kostenfrei verfügbar sind.

Die wichtigsten technologischen Entscheidungen im Bereich Software und Hardware sind:

\begin{itemize}
    \item \textbf{Entwicklungsumgebung (IDE):} Als integrierte Entwicklungsumgebung kommt die IntelliJ IDEA Ultimate Edition zum Einsatz.
    
    \item \textbf{Build-Management:} Die Verwaltung der Projektabhängigkeiten und der Build-Prozess erfolgen über Apache Maven.
    
    \item \textbf{Zielsystem:} Die Anwendung ist als Windows-Desktop-Anwendung konzipiert.
    
    \item \textbf{Auslieferung:} Das Projekt muss als einzelne, ausführbare Windows-Datei (.exe) bereitgestellt werden, die ohne Installation auf den Schulrechnern lauffähig ist. Die Kompatibilität mit der schulischen IT-Infrastruktur ist zwingend erforderlich.
    
    \item \textbf{Hardware:} Die Entwicklung erfolgt auf einem vorhandenen Entwicklungsrechner. Die Performance muss jedoch auch bei Belastungstests mit bis zu 2.000.000 simulierten Wählern stabil gewährleistet sein.
\end{itemize}


\subsubsection{Programmiersprache und Frameworks}

Die technische Kernumsetzung basiert auf einem robusten und für Desktop-Anwendungen optimierten Technologie-Stack:

\begin{itemize}
    \item \textbf{Programmiersprache:} Als Programmiersprache wird Java \ac{SE} (Standard-Edition) verwendet.
    
    \item \textbf{Java Development Kit (\ac{JDK}):} Konkret wird das \ac{JDK} 25 in der \ac{LTS} (Long Term Support)-Version eingesetzt.
    
    \item \textbf{\ac{GUI}-Framework:} Für die grafische Benutzeroberfläche (\ac{GUI}) und die Echtzeit-Visualisierung wird das Framework JavaFX genutzt.
\end{itemize}


\subsection{Implementierung}

\subsubsection{Kernkomponenten der Simulation}

% TODO: Hier die Kernkomponenten beschreiben
[Hier beschreiben: Welche Klassen/Module gibt es? Wie interagieren sie?]

% Beispiel für Code-Listing:
% \begin{lstlisting}[language=Java, caption={Beispiel: SimulationEngine Klasse}]
% public class SimulationEngine {
%     private List<Voter> voters;
%     private List<Party> parties;
%     
%     public void runSimulationStep() {
%         // Simulationslogik
%     }
% }
% \end{lstlisting}


\subsubsection{Zufallselemente und Verteilung}

% TODO: Hier die Zufallsverteilungen beschreiben
[Hier beschreiben: Normalverteilung, Gleichverteilung, Exponentialverteilung -- Wo und wie werden sie eingesetzt?]


\subsection{Benutzeroberfläche}

\subsubsection{UI-Konzept und Usability}

% TODO: Hier das UI-Konzept beschreiben
[Hier beschreiben: Wie ist die Oberfläche aufgebaut? Wie wird ISO 9241-110 umgesetzt?]


\subsubsection{Elemente und Animation}

% TODO: Hier die Animation beschreiben
[Hier beschreiben: Welche Animation wurde implementiert? Wie unterstützt sie die UX?]

\newpage

% ============================================================
% KAPITEL 4: QUALITÄTSSICHERUNG
% ============================================================

\section{Qualitätssicherung}

\subsection{Testkonzept}

% TODO: Hier das Testkonzept beschreiben
[Hier beschreiben: Welche Testarten werden eingesetzt? (Unit-Tests, Integrationstests, Systemtests, Usability-Tests)]

% Beispiel für eine Testübersicht:
% \begin{table}[H]
%     \centering
%     \caption{Übersicht der Testarten}
%     \begin{tabularx}{\textwidth}{|l|X|l|}
%         \hline
%         \textbf{Testart} & \textbf{Beschreibung} & \textbf{Werkzeug} \\
%         \hline
%         Unit-Tests & Test einzelner Methoden und Klassen & JUnit 5 \\
%         \hline
%         Integrationstests & Test der Komponenteninteraktion & JUnit 5 \\
%         \hline
%         Systemtests & Test des Gesamtsystems & Manuell \\
%         \hline
%     \end{tabularx}
% \end{table}


\subsection{Testdurchführung}

% TODO: Hier die Testdurchführung beschreiben
[Hier beschreiben: Welche Tests wurden durchgeführt? Was waren die Ergebnisse?]


\subsection{Validierung der Zufallsmodelle}

% TODO: Hier die Validierung beschreiben
[Hier beschreiben: Wie wurde sichergestellt, dass die Zufallsverteilungen korrekt implementiert sind?]


\subsection{Soll-Ist-Vergleich}

% TODO: Hier den Soll-Ist-Vergleich einfügen
[Hier beschreiben: Vergleich der geplanten mit den tatsächlichen Zeiten]

% Beispiel für Soll-Ist-Vergleich:
% \begin{table}[H]
%     \centering
%     \caption{Soll-Ist-Vergleich der Projektphasen}
%     \begin{tabularx}{\textwidth}{|X|r|r|r|}
%         \hline
%         \textbf{Phase} & \textbf{Soll (h)} & \textbf{Ist (h)} & \textbf{Differenz} \\
%         \hline
%         Projektplanung & 20 & 22 & +2 \\
%         \hline
%         Implementierung & 60 & 58 & -2 \\
%         \hline
%         Test & 25 & 25 & 0 \\
%         \hline
%         Dokumentation & 15 & 15 & 0 \\
%         \hline
%         \textbf{Gesamt} & \textbf{120} & \textbf{120} & \textbf{0} \\
%         \hline
%     \end{tabularx}
% \end{table}

\newpage

% ============================================================
% KAPITEL 5: ZUSAMMENFASSUNG UND AUSBLICK
% ============================================================

\section{Zusammenfassung und Ausblick}

\subsection{Fazit}

% TODO: Hier das Fazit schreiben
[Hier beschreiben: 
- Wurden die Projektziele erreicht?
- Was lief gut, was könnte verbessert werden?
- Persönliche Erkenntnisse aus dem Projekt
- Beantwortung der Fragestellung: "Wie beeinflussen externe Faktoren das Endergebnis einer demokratischen Wahl?"]


\subsection{Ausblick}

% TODO: Hier den Ausblick schreiben
[Hier beschreiben:
- Mögliche Erweiterungen der Anwendung
- Verbesserungspotenziale
- Übertragbarkeit auf andere Anwendungsfälle]

\newpage

% ============================================================
% 6. LITERATURVERZEICHNIS / QUELLENANGABEN
% ============================================================
\section*{Literaturverzeichnis}
\addcontentsline{toc}{section}{Literaturverzeichnis}

% Wenn du BibLaTeX nutzt:
% \printbibliography[heading=none]

% Oder manuell:
[Hier kommen deine Quellen hin]

\newpage

% ============================================================
% 7. ANLAGEN
% Römische Nummerierung, beginnend bei I
% ============================================================
\pagenumbering{Roman}
\appendix

\section*{Anlagen}
\addcontentsline{toc}{section}{Anlagen}

% ============================================================
% ANHANG
% ============================================================

\subsection*{A1 UML-Diagramme}

% --- USE-CASE-DIAGRAMM ---
\begin{figure}[H]
    \centering
    % Ersetze "usecase.png" mit deinem Dateinamen
    % \includegraphics[width=0.8\textwidth]{bilder/usecase.png}
    
    [Hier Bild einfügen: usecase.png]
    
    \caption{Anwendungsfalldiagramm des Wahlsystems}
    \label{fig:usecase}
\end{figure}

% --- KLASSENDIAGRAMM ---
% \begin{figure}[H]
%     \centering
%     \includegraphics[width=\textwidth]{bilder/klassendiagramm.png}
%     \caption{Klassendiagramm}
%     \label{fig:klassendiagramm}
% \end{figure}

% --- SEQUENZDIAGRAMM ---
% \begin{figure}[H]
%     \centering
%     \includegraphics[width=\textwidth]{bilder/sequenzdiagramm.png}
%     \caption{Sequenzdiagramm: Datenfluss}
%     \label{fig:sequenz}
% \end{figure}


\newpage
\subsection*{A2 Zeitplanung}

\begin{figure}[H]
    \centering
    % Ersetze "gantt.png" mit deinem Dateinamen
    % \includegraphics[width=\textwidth]{bilder/gantt.png}
    
    [Hier Bild einfügen: gantt.png]
    
    \caption{Gantt-Diagramm}
    \label{fig:gantt}
\end{figure}


\newpage
\subsection*{A3 Screenshots der Anwendung}

% --- SCREENSHOTS ---
% \begin{figure}[H]
%     \centering
%     \includegraphics[width=0.9\textwidth]{bilder/screenshot_main.png}
%     \caption{Hauptansicht der Anwendung}
%     \label{fig:screenshot_main}
% \end{figure}

[Hier Screenshots der fertigen Anwendung einfügen]


\newpage
\subsection*{A4 Quellcode-Auszüge}

% Falls du wichtige Code-Ausschnitte zeigen willst:

% \begin{lstlisting}[language=Java, caption={Kernmethode der Simulation}]
% public void runSimulation() {
%     while (simulationRunning) {
%         calculateVoterMovement();
%         applyRandomEvents();
%         updateStatistics();
%         notifyObservers();
%     }
% }
% \end{lstlisting}

[Hier wichtige Code-Auszüge einfügen]


\end{document}